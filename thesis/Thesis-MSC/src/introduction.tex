%----------------------------------------------------------------------------
\chapter*{Introduction}\addcontentsline{toc}{chapter}{Introduction}
%----------------------------------------------------------------------------

Modeling is an important tool of engineering problem solving, which aims at
reducing problem complexity and designing possible problem solution. The
paradigm of \emph{Model Driven Engineering} (MDE) lies on these principles. It
is intended to create a system model from a high level domain specific model
through various transformation steps, from which it is possible to generate the
required application's source code and documentation.

One of the most important tasks during MDE is doing queries on the object
structures defined by the models. This is useful for several steps during
development, like transformation, source code generation, testing or generating
documentation.

Nowadays a new branch of MDE is spreading called Runtime Modeling which focuses
on using the abstract system model not only during design time, but also
integrated in the running application. This approach allows the system to
support intelligent functions, such as, in the case of Cyberphysical Systems,
responding to environmental events, inference and intervention based on system
state. This new branch of Runtime Modeling however poses new problems. Studies show
that most modern techniques are not suitable for (soft) real time environments,
nor for the small resources of embedded systems.

My job during this thesis is to give a solution to this problem via
extending an already existing Java based model querying solution to be usable in
such environments with the usage of \CPP{}.

%----------------------------------------------------------------------------
\section*{Document structure}\label{sec:DocumentStructure}
%----------------------------------------------------------------------------

The document is composed of the following main chapters:

\begin{itemize}
  \item Chapter 1. focuses on showing the technologies related to the developed program.
  \item Chapter 2. gives an overview of the local search model querying technique
  and its main concepts.
  \item Chapter 3. explains the designed application architecture and the main
  design decisions, difficulties.
  \item Chapter 4. walks through the workflow with the developed program with the help of a practical example.
  \item Chapter 5. shows the results of scalability and performance evaluation.
  \item Chapter 6. concludes the thesis reflecting on the issues encountered
  during development and assessing the final product.
\end{itemize}