\chapter{Overview of Approach}

This chapter will mostly focus on the main design decisions and the architecture
of the application.

%----------------------------------------------------------------------------
\section{Architecture}\label{sec:Architecture}
%----------------------------------------------------------------------------

The application's componetns can be separated into three main categories:

\begin{itemize}
  \item Java-Compile Time - components that are used while designing the
  queries and generating code.
  \item \CPP{} Compile Time - the generated code.
  \item \CPP{} Runtime - the code running the queries.
\end{itemize}

The architecture of the application is shown on figure \figref{architecture}.
The main component of the architecture is the abstract model. This model
serves as the description of the possible object types and their relationships.
In my case this is basically similar to an UML class diagram. Further
detail will be discussed in chapter \sectref{CppObjectModel}.

\begin{figure}[!ht]
\centering
\includegraphics[width=150mm, keepaspectratio]{figures/architecture.png}
\caption{The architecture of the engine.}
\label{fig:architecture}
\end{figure}

From the abstract model, the POCO generator (\sectref{GeneratedCodeStructure}) generates the
\CPP{} classes. These are the classes the runtime instantiates and the queries
can run on.

To write queries, a query editor is supplied with the application. This editor
allows writing the queries with content assist and refactoring support. The
pattern language is similar to the EMF-IncQuery pattern language, the main
difference is the model it works on is not an EMF metamodel, but the abstract
model defined by the user.

From the query definition the search graph (called \emph{PSystem} in
EMF-IncQuery) gets transformed. This search graph contains every information
necessary to create the search plan. The search plan gets transformed from the
search graph in java at first, then based on this plan the actual search
operations get transformed into \CPP{} code. The available search operations and
their execution is coded in a runtime library. The assembly of the actual search
plan and it's execution gets obfuscated by some generated code, the user
only has to call the appropriate generated method to get the matching elements
to a specific query.

This architecture makes it so that the user only has no interaction with any
localsearch related part of the application, i.e. he has to create the model,
write the queries and later on call the queries in his \CPP{} application,
everything else is handled hidden from him.
