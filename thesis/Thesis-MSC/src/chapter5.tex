\chapter{Scaleability and Performance}

One of the most important motivating factor for writing a \CPP{} implementation
of the local search runtime was performance. The assumption was that the \CPP{}
implementation, even though it might have many restrictions, would be
considerably faster than the original Java one. Several performance benchmarks
were done throughout the development to verify or deny the previous statement.

\section{Model}

The model used during the performance benchmarks was an instance of the
previously introduced school metamodel (figure \figref{School_Metamodel}).
Table \tabref{model_size} shows the model scales used during performance
benchmarking.

\begin{table}[ht]
	\footnotesize
	\centering
	\caption{Model size}\label{tab:model_size}
	\begin{tabular}{ | l | c | c | c | c | c | c |}
	\hline
	Scale	& Model Elements	& References	\\ \hline
	1 		&  2\,040			& 12\,633		\\
	2 		&  4\,080			& 25\,266		\\
	4		&  8\,160			& 50\,532		\\
	8		&  16\,320			& 101\,064		\\	
	16 		&  32\,640			& 202\,128		\\	
	32		&  65\,280			& 404\,256		\\	
	64 		&  130\,560			& 808\,512		\\	
	128 	&  261\,120			& 1\,617\,024	\\	
	256 	&  522\,240			& 3\,234\,048	\\	
	512		&  1\,044\,480		& 6\,468\,096	\\
	1\,024	&  2\,088\,960		& 12\,936\,192	\\	
	\hline
	\end{tabular}
	\label{tab:TabularExample}
\end{table}

The model size is defined with scale, which is essentially a simple multiplier
of the number of instances of each class defined in the metamodel. Larger
scales of a model simply created new instances of schools with similar
internal structure, but without any relationships not between the schools. The
instance model of a specified scale was generated using a deterministic
algorithm which was implemented in the same way in both Java and \CPP{},
allowing the comparison of the two framework implementation.

The measurements compare the performance of three implementations: Java Local
Search, Java RETE\cite{EIQ-Rete} and C++ iterator based. The Java Local Search
based implementation is the Java implementation of the local search algorithm
described in this thesis. The search plan generator is the same for both
the Java and \CPP{} Local Search solutions, but the Java implementation has
access to the model statistics which can influence the search plans. The Java
RETE algorithm based implementation is the default pattern matching
implementation used by \EIQ{}, which also means that this is the most mature of
the three approaches. It allows for incremental pattern matching, which
requires extensive caching which results in more memory usage and slower
initial search, but the cache is kept up to date internally through
notifications from the model, which means the second time a query is run it
will have a constant, very small return time.

\section{Performance measurements}

The first measurements focus on the time it takes to initialize and execute a
query on a model, retrieving every match. The first pattern is the really simple
\emph{students} pattern \listref{meas_students}, which retrieves matches every
student in the model. The main difficulty for this pattern is that the result
set is very large (more than 2 million elements for the largest scale), this
means the main bottleneck is the creation of the result set itself.

\begin{lstlisting}[frame=single,float=!ht,language=IQPL,
label=listing:meas_students, caption=The students pattern]
pattern students(student) {
	Student(student);
}
\end{lstlisting}

Figure \figref{meas_students} shows the execution times measured for each platform
with different implementations and on three model sizes: 256, 512 and 1024. Java
LS refers to the Java Local Search implementation, while the rest is self
explanatory.

\begin{figure}[!ht]
\centering
\includegraphics[width=120mm, keepaspectratio]{figures/meas_students.png}
\caption{Performance of the \emph{students} query}
\label{fig:meas_students}
\end{figure}

The measured times show the \CPP{} implementation performing considerably
better, then the Java ones. In the case of such a simple pattern this was
expected, as the \CPP{} implementation does a lot less preprocessing and most of
the query time consists of the assembly of the result set. The surprising part is
the fact the Java implementation of the local search algorithm does slightly
worse than the RETE based algorithm, which does a lot of advanced preprocessing and
caching. This could be the result of the immature codebase for the local search
based implementation, as it is still in development and might have optimization
issues.

The next examined pattern is the \emph{classesOfTeacher} pattern
(\listref{meas_classesOfTeacher}). This pattern is slightly more complex than
the previous one, however the result set is really small. The main takeaway from
this benchmark is how fast an implementation can respond to a very simple query
with only a few thousand matches.

\begin{lstlisting}[frame=single,float=!ht,language=IQPL,
label=listing:meas_classesOfTeacher, caption=The classesOfTeacher pattern]
pattern coursesOfTeacher(course, teacher) {
	Teacher.courses(teacher, course);
}

pattern classesOfTeacher(teacher, schoolClass) {
	find coursesOfTeacher(course, teacher);
	Course.schoolClass(course, schoolClass);
}
\end{lstlisting}

The following chart (\figref{meas_classesOfTeacher}) shows the results of this
benchmark. In this case the difference between the Java and \CPP{}
implementations is absolutely massive. The most probable reason is the
preprocessing the Java implementations do on the model, which results in the
terrible return time. The Java local search implementation lags behind even more
than the last time, again, likely because of the not yet optimized code.

\begin{figure}[!ht]
\centering
\includegraphics[width=120mm,
keepaspectratio]{figures/meas_classesOfTeacher.png}
\caption{Performance of the \emph{classesOfTeacher} query}
\label{fig:meas_classesOfTeacher}
\end{figure}

The next pattern is the \emph{studentsOfSchool}
(\listref{meas_studentOfSchool}), which is a moderately complex pattern. The
complexity comes from the fact that it is not possible to navigate through the
associations as described in the pattern because of the direction the
associations can be navigated in, thus the search plan has to overcome this via
starting from the center of the navigation chain and going in both directions
from there.

\begin{lstlisting}[frame=single,float=!ht,language=IQPL,
label=listing:meas_studentOfSchool, caption=The studentOfSchool pattern]
pattern studentsOfSchool(student, school) {
	Student.schoolClass.courses.school(student, school);
}
\end{lstlisting}

The more advanced pattern means the gap between the \CPP{} and Java
implementations became slightly smaller as seen on figure
\figref{meas_studentOfSchool}. As seen before, the Java local search
implementation still lags behind the RETE implementation.

\begin{figure}[!ht]
\centering
\includegraphics[width=120mm,
keepaspectratio]{figures/meas_studentOfSchool.png}
\caption{Performance of the \emph{studentOfSchool} query}
\label{fig:meas_studentOfSchool}
\end{figure}

The next pattern in line is the \emph{mutualFriendsInSchool}. This pattern is
the most complex of all the benchmarked patterns. The pattern searches for
two students in a single school whose friends with relationship is mutual. This
pattern is an absolute worst case scenario, as it calls an already complex
pattern twice resulting in a huge join at as the first steps of the search plan.

\begin{lstlisting}[frame=single,float=!ht,language=IQPL,
label=listing:meas_mutualFriendsInSchool, caption=The mutualFriendsInSchool pattern]
pattern mutualFriendsInOneSchool(studentA, studentB) {
	find studentsOfSchool(studentA, school);
	find studentsOfSchool(studentB, school);
	Student.friendsWith(studentA, studentB);
	Student.friendsWith(studentB, studentA);
}
\end{lstlisting}

In the case of this pattern the run times were a lot slower as expected,
which is why the times are written in milliseconds instead of microseconds
while the model scales are much smaller. The results, as seen on figure
\figref{meas_mutualFriendsInSchool}, do not continue the trend of the \CPP{}
version being faster than the Java implementation. In this case, both \CPP{}
implementations are slower then the Java RETE solution, while being slightly
faster than the Java local search version. The most likely reasons is a sub
optimal search plan, which more than likely significantly hurt the performance
of all local search based solutions.

\begin{figure}[!ht]
\centering
\includegraphics[width=120mm,
keepaspectratio]{figures/meas_mutualFriendsInSchool.png}
\caption{Performance of the \emph{mutualFriendsInSchool} query}
\label{fig:meas_mutualFriendsInSchool}
\end{figure}

The following figure (\figref{meas_second_run}) shows the times measured when
running the queries a second time. The pattern used in this benchmark is the
\emph{classesOfTeacher} (\listref{meas_classesOfTeacher}). In this case, the
\CPP{} implementations performed identical to the first run. The Java local
search based version shows slight massive improvements in its speed. The reason
for the speedup is that no preprocessing is necessary this time. The most
significant upgrade in speed is in the case of the Java RETE algorithm based
solution, as the times dropped to constant 1 ms independent from the model size.
This is because of the caching and incremental updating this algorithm is
capable of.

\begin{figure}[!ht]
\centering
\includegraphics[width=120mm,
keepaspectratio]{figures/meas_second_run.png}
\caption{Performance of the second run of a query}
\label{fig:meas_second_run}
\end{figure}

The last figure (\figref{meas_runtime_vs_iterator}) demonstrates the
performance difference between the iterator based solution and the runtime based
one using the \emph{studentsOfSchool} pattern. The two solutions are qute close
to each other, they stay within 5-10\% of each other on most measurements with
the iterator based solution being the faster one. This is most likely because
it is not necessary to select the next search operation from an array as the
search operations are hard coded in the code.


\begin{figure}[!ht]
\centering
\includegraphics[width=120mm,
keepaspectratio]{figures/meas_runtime_vs_iterator.png}
\caption{Performance comparison of the runtime and iterator based implementations}
\label{fig:meas_runtime_vs_iterator}
\end{figure}
